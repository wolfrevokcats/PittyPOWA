\documentclass[a4paper, 12pt, twoside, openright, fleqn]{book}

% language settings
\usepackage[italian]{babel}
\usepackage[T1]{fontenc}
\usepackage[utf8]{inputenc}
\usepackage{fancyhdr}
\usepackage{subcaption}
\usepackage[usenames, dvipsnames, table, tikz]{xcolor}
\usepackage{tikz,float,pdfpages,pgfplots,listing, steinmetz}
\usepackage{graphicx,graphics,wrapfig,changepage}

% math package
\usepackage{mathtools,amsthm,environ,cancel,cases}
\usepackage{amsmath, amssymb, dsfont, bm,blkarray}
\DeclareMathOperator*{\argmin}{arg\,min}
\DeclareMathOperator*{\argmax}{arg\,max}
\DeclarePairedDelimiter{\ceil}{\lceil}{\rceil}

\graphicspath{{./img/} {./.tmp/} }
% tikz figures
% \usepackage{tikz}
\usetikzlibrary{shapes, arrows, automata,circuits.ee.IEC}
\tikzset{blk/.style={draw, minimum size=0.5cm, text width = 1.8 cm}}
\tikzset{chan/.style={cylinder,shape aspect = 0.2,draw, minimum size=0.5cm, text width = 1.5 cm}}
% add support for url and cross-references in PDF output
\usepackage{url}
\renewcommand{\UrlFont}{\color{black}\small\ttfamily}
\usepackage[colorlinks=true, linkcolor=black, citecolor=black, urlcolor=black]{hyperref}

% support for glossary and acronyms
\usepackage[acronym]{glossaries}
% \newacronym{geraf}{GeRaF}{Geographic Random Forwarding}
% Costumization of theorem style
\newtheoremstyle{theoremdd}% name of the style to be used
  {\topsep}% measure of space to leave above the theorem. E.g.: 3pt
  {\topsep}% measure of space to leave below the theorem. E.g.: 3pt
  {\itshape}% name of font to use in the body of the theorem
  {0pt}% measure of space to indent
  {\bfseries}% name of head font %\color{Mahogany}
  { -- }% punctuation between head and body
  { }% space after theorem head; " " = normal interword space
  {\thmname{#1}\thmnumber{ #2}\thmnote{ (#3)}}

\theoremstyle{theoremdd}
% support for counting
\newtheorem{theorem}{Theorem}[section]
\newtheorem{corollary}{Corollary}[theorem]
\newtheorem{lemma}[theorem]{Lemma}
\newtheorem{definition}{Definition}
\newtheorem{remark}{Remark}
\newtheorem{example}{Example}


\newenvironment{exercize}[2][Exercize]{\begin{trivlist}
\item[\hskip \labelsep {\bfseries #1} \hskip \labelsep {\bfseries #2}]}{\end{trivlist}}
\newenvironment{solution}[2][Solution]{\begin{trivlist}
\item[\hskip \labelsep {\bfseries #1}\hskip \labelsep {\bfseries #2}]}{\end{trivlist}}
%If you want to title your bold things something different just make another thing exactly like this but replace "problem" with the name of the thing you want, like theorem or lemma or whatever
\newcounter{exercize}
\setcounter{exercize}{1}


% added support for proof parts
\theoremstyle{remark}
\newtheorem{proofpart}{Solution}
\renewcommand\theproofpart{\Roman{proofpart}}
\makeatletter
\@addtoreset{proofpart}{theorem}
\makeatother

% specific modification to basic book template of our book document
% \renewcommand{\chaptername}{Section}
% \addto\captionsenglish{\renewcommand{\chaptername}{Section}}

% \renewcommand\qedsymbol{\includegraphics[width=1.5cm]{occhiali}}
%\renewcommand\qedsymbol{$\square$\itshape QED}

%split and equation environment
\NewEnviron{esp}{%
\begin{equation}\begin{split}
  \BODY
\end{split}\end{equation}
}

\NewEnviron{esp*}{%
\begin{equation*}\begin{split}
  \BODY
\end{split}\end{equation*}
}


%------------------------------ define Abstract environment, missing in the 'book' class
\newenvironment{abstract}{\cleardoublepage \null \vfill \begin{center}\bfseries\abstractname \end{center}}{\vfill\null}
\addto\captionsenglish{\renewcommand*\abstractname{Sommario}} % change Abstract title
%------------------------------ active url
\usepackage{url}
%\usepackage{svg}
\usepackage{ifluatex}
\renewcommand{\UrlFont}{\color{black}\small\ttfamily}

% \usepackage[colorlinks=true, linkcolor=black, citecolor=black, urlcolor=black]{hyperref} % active ref

%------------------------------ macros
\newcommand{\sectionname}{Section} % define Section ref
\newcommand{\subsectionname}{Sub-section} % define Sub-section ref
\renewcommand*\arraystretch{1.4} % tables padding
\newcommand{\N}{\mathcal{N}}
\DeclareInputText{176}{°}

% Useful Aliases
\def\beq{\begin{equation}}
\def\eeq{\end{equation}}
\def\bal{\begin{align}}
\def\eal{\end{align}}
\def\prob{\ensuremath\mathbb{P}}
\def\exp{\ensuremath\mathbb{E}}
\def\pois{\mathcal P}
\newcommand{\Hb}{\mathbb{H}}
\newcommand{\Sb}{\mathbb{\Sigma}}
\newcommand{\U}{\mathbb{U}}
\newcommand{\F}{\mathbb{F}}
\newcommand{\V}{\mathbb{V}}
\newcommand{\A}{\mathbb{A}}
\newcommand{\B}{\mathbb{B}}
\newcommand{\C}{\mathbb{C}}
\newcommand{\E}{\mathbb{E}}
\newcommand{\I}{\mathbb{I}}
\newcommand{\Y}{\mathbb{Y}}
%\newcommand{\N}{\mathbb{N}}
\newcommand{\W}{\mathbb{W}}
\newcommand{\Z}{\mathbb{Z}}
\newcommand{\R}{\mathbb{R}}
\newcommand{\X}{\mathbb{X}}
\renewcommand{\Pr}{\mathbb{P}} %probability
\renewcommand{\P}{\mathcal{P}} % distribution
\newcommand{\Q}{\mathbb{Q}}
\newcommand{\D}{\mathbb{D}}
\newcommand{\Rm}{\mathbb{R^{-1}}}
\newcommand{\J}{\mathbb{J}}
\newcommand*{\Chi}{\mbox{\Large $\chi$}}% big chi
\newcommand{\Fig}[1]{Fig.~\ref{#1}}
\newcommand{\eq}[1]{(\ref{#1})}
\newcommand{\Tab}[1]{Tab.~\ref{#1}}
\newcommand{\Sec}[1]{Sec.~\ref{#1}}
\newcommand{\indep}{\mathrel{\perp\mspace{-10mu}\perp}}
\newcommand{\RN}[1]{ \textup{\uppercase\expandafter{\romannumeral#1}}} %This is for inserting Roman Numbers (useful for \stackrel in equations so that you don't confuse indicating numbers from equation numbers

\begin{document}
\frontmatter

\begin{titlepage} %------------------------------ TITLE PAGE
\begin{center}

\hspace{0.5cm}

\emph{\Large{TELEMEDICINE}} \\
\vspace{1cm}
% \includegraphics[width=9cm]{img/bella.png}\\
\vspace{0.5cm}
{written with love by\par}
{\Large Matilde Boschiero and Andrea Pittaro\par}
\end{center}

\vfill
\begin{center}
\noindent\makebox[\linewidth]{\rule{\textwidth}{0.4pt}}
\textsc{Academic Year 2017/2018}
\end{center}
\end{titlepage}

\begingroup %------------------------------ CONTENTS
  \makeatletter
  \let\ps@plain\ps@empty
  \makeatother
  \tableofcontents
  \clearpage
\endgroup
\mainmatter
\chapter{Review on signals}

lez lunedì

\section{Random Processes (RPs)}
All the signals can be modeled as random processes, so we need to review the continuous $rp_s$ and the discrete ones

\begin{esp}
  x: \quad &\R \times \Omega \to \R \\
  x:  \quad& \Z \times \Omega \to \R
\end{esp}

grafico t0

we can determinate the rp giving its probability density function(PDF), in particular we define some property of it:
\begin{itemize}
  \item mean $m_x(t)=\E[x(t)]$
  \item statistical power $M_x(t)=\E[|x(t)|^2]$
  \item autocorrelation function $r_x(t,\tau)=\E[x(t)\cdot x^*(t-\tau)]$
\end{itemize}
One particular property of the autocorrelation function is that $r_x(t,0)=M_x(t)$

\section{Stationary processes}
A r.p. is said to be stationary with respect to a statistical description(e.g. mean, autocorrelation) if such desctiption is invatiant to any time sgift. That is all $rp_s \, x_1(t) = x(t-t_{_0})$ with arbitrary$t_0$ have the same statistical description as x(t).

\subsection{Wide Sense Statistical process (WSS)}
$x(t,\omega)$ is stationary both in mean and in autocorrelation $\implies r_x(t,\tau) = r_x(\tau)$, which means that the time dependency drops

\subsection{Cyclostationary processes}
A r.p. is said to be cyclostationary if that description (***) is invariant to a time sgift of the signal by some quantity $T_c$ that is called the \emph{cyclostationary period}.

\subsection{Ergodic processes}
A stationary random process (in mean) is said to be ergodic in mean if the time average converges to its statistical mean. In some sense
\begin{equation}
  \frac{1}{2 u} \int_{-u}^u x(\omega,t) \partial t \stackrel{u \to +\infty}{\rightarrow} m_x \forall \omega \in \Omega
\end{equation}

\subsection{Cross correlation function}
Given a pair of $rp_s$ $x(t)$ and $y(t)$, we have that
\begin{equation}
  r_{xy}(t,\tau) = \E[x(t) \cdot y^*(t-\tau)]
\end{equation}
in particular, $\forall t,\tau$
\begin{equation}
  \begin{cases}
    \text{if } r_{xy}(t,\tau)=0 & \text{ the two processes are \textbf{ORTHOGONAL}}\\
    \text{if } r_{xy}(t,\tau)=m_x \cdot m_y^* & \text{ the two processes are \textbf{UNCORRELATED}}\\

  \end{cases}
\end{equation}

\subsection{Power spectral density (PSD)}
Let $x(t)$ be WSS, then we define the PSD of $x(t)$ as the Fourier transform of the autocorrelation function.
\begin{equation}P_x(f)=
  \begin{cases}
    \int_{-\infty}^{+\infty} r_x(\tau)e^{-\jmath 2\pi f \tau} \partial \tau \\
    \sum\limits_{k=-\infty}^{+\infty} r_x(k T)e^{-\jmath 2\pi f k T}
  \end{cases}
\end{equation}
\subsection{PSD properties}
Some important properties of the PSD are
\begin{enumerate}
  \item $P_x(f)\in \R \forall f$
  \item $P_x(f)\ge 0 \forall f$
  \item Its integral gives $M_x$
  \item For real valued processes $P_x(f)=P_x(-f)$
\end{enumerate}


We now define the band of the signal based on the PSD:

\begin{description}
  \item[FULLBAND] of a WSS r.p. $x(t)$ is the support of its PSD, which means
  $$\hat{\B_x} = \lbrace f \in \R : P_x(f)>0\rbrace$$
  \item[BAND] of a real valued WSS r.p. $x(t)$ is the portion of the support of the positive frequencies.
  $$\hat{\B_x} = \lbrace f>0 : P_x(f)>0\rbrace$$
\end{description}

\section{White random processes}
A WSS r.p. $x(t)$, whose PSD is constant over the whole frequencies axis ($P_x(f) = P_0 \forall f \in \R$)
\begin{equation}
  \implies
  \begin{cases}
    r_x(\tau) = P_0 \cdot \delta(\tau) \\
    r_x(m) = \frac{P_0}{T} \cdot \delta(m)
  \end{cases}
\end{equation}
\subsection{Gaussian processes}
A real valued rp is said to be Gaussian if all the $r.v._s$ that we can extract from the overall r.p., taking an arbitrary number of variables are gaussian vectors.
\begin{equation}
  x(t)\sim \mathcal{N} (m_x(t),\sigma^2_x(t)) \quad \text{where } \sigma_x(t) =  \sqrt{M_x(t)-m_x^2(t)}
\end{equation}
We also derive that the probability that the $x(t)$ is between b and c as
\begin{equation}
  \mathbb{P}[b<x(t)<c] = Q\left(\frac{b-m_x(t)}{\sigma_x^2(t)}\right)-Q\left(\frac{c-m_x(t)}{\sigma_x^2(t)}\right)
\end{equation}

\begin{figure} \centering
  % \documentclass{standalone}
% \usepackage{tikz}
% \usetikzlibrary{automata,arrows,circuits.ee.IEC}
% \tikzset{blk/.style={draw, minimum size=0.5cm, text width = 1.8 cm}}
% \tikzset{chan/.style={cylinder,shape aspect = 0.2,draw, minimum size=0.5cm, text width = 1.5 cm}}
% \begin{document}
  \begin{tikzpicture}[circuit ee IEC]
    \node[blk] at (-3,0) (a-al) {Anti aliasing filter};
    \node at (1,0.5) (c) {};
    \node[blk,text width = 0.5cm] at (1.5,0) (q) {Q};
    \node[blk] at (5,0) (ibmap) {IBMAP};
    \node[blk,text width = 0.5cm] at (8,0) (ps) {P/S};
    \node[] at (0,0) (sampl) {   };

    \draw (-1.5,0) to [make contact={info={$a(nT_s)$},info'={$T_s$}}] (1,0);
    \draw[-stealth] (-0.3,0.3) to [bend left] (-0.1,-0.2);
    \draw[-stealth] (-5,0) -- node[above left]{$a(t)$} (a-al);
    \draw[-stealth] (a-al) -- node[above]{$a(t)$} (-0.5,0);
    \draw[-stealth] (0.3,0) -- (q);
    \draw[-stealth] (q) -- node[above]{$a_q(n T_s)$} (ibmap);
    \draw[-stealth] (ibmap) -- node[above]{$c(n T_s)$} (ps);
    \draw[-stealth] (ps) -- node[above]{b} (9,0);
  \end{tikzpicture}
% \end{document}

  \caption{The scheme for an ADC}
  \label{}
\end{figure}
\section{Quantizer}
Let the input signal be $a(n T_x) \in \R$, the intermediate output(1) is $a_q(nT_s) \in A_q = \lbrace q_0,\dots,q_{L-1} \rbrace$ with L the number of levels of the quantizer.
The intermediate output $\hat{c}(n Ts)$ is the codewords, which means a sequence of bits. The final output is a bit stream $\lbrace b_l \rbrace$ with a bit time $T_b$

\subsection{Uniform quantizer}
An uniform quantizer has L equally spaced levels represented by $b = log_2(L)$ bits. We define
\begin{itemize}
  \item $\Delta$ as the quantization step size
  \item the threshold $v_i$, $i = 1, 2, \dots, L - 1$, with $v_0 = -\infty \ ,\ v_L = \infty$
  \item $v_{sat}$ as the saturation signal, which is the intensity over which the signal is always captured as one of the two extremal thresholds.
  \item $\Lambda_q = \frac{\E[a^2(nTs)]}{\E[e_q^2(nTs)]}$ the SNR of the quantizer.
  \item $e_q = a_q(n Ts) - a(n Ts)$ the quantization error which can be divided into
  \begin{itemize}
    \item $e_{sat}$ the error made in the saturation zone. It can be neglegible if we set the quantizer levels such that the signal goes rarely into the saturation region
    \item $e_{gr}$ the quantization error in the granular region, called usually granular error, which is usually the main component.

    In uniform quantizer $e_{gr}\sim \U\left(\left[-\frac{\Delta}{2};\frac{\Delta}{2}\right]\right)$
    $\implies \Lambda_q = \frac{M_a}{M_q} \approx \frac{12 \sigma_a^2 }{\Delta^2}$
  \end{itemize}
\end{itemize}

The P/S, parallel to serial converter, converts the codewords into bitstreams. Hence, the bit frequency is higher and the bit period lower ($Tb = \frac{Ts}{b}$)

Moreover we define the symbol rate as $F=\frac{1}{T} = \frac{R_b}{\log_2(M)}$

\begin{figure}\centering
  % \documentclass{standalone}
% \usepackage{tikz}
% \usetikzlibrary{automata,arrows}
%
% \begin{document}
  \begin{tikzpicture}
    \node[blk, text width = 2cm] at (-2,0) (b) {Digital demodulator};
    \node[chan] at (2,0) (q) {Channel};

    \draw[-stealth] (-4.5,0) -- node[above]{$a_k$} (b);
    \draw[-stealth] (b) -- node[above] {$s_{TX}(t)$} (q);
  \end{tikzpicture}
% \end{document}

  \caption{a digital modulator scheme}
  \label{}
\end{figure}
\subsubsection{Modulations}
We now introduce the PAM (Pulse Amplitude Modulation) modulation, where each transmitted symbol is defined as the symbol with this map: $a_k \to s_{a_k} (t) = a_k \cdot h_{tx}(t)$. $h_{tx}$ is defined as the impulse shape, usually taking the form of the rect or the raised cosine (ircos).
The general transmission can be shown as
\begin{equation}
  s_{tx}(t) = \sum\limits_{k=-\infty}^{+\infty} a_k \, h_{tx}(t - kT)
\end{equation}

at the receiver we obtain
\begin{equation}
  r(t) = s_{RC}(t) + w_{RC}(t) = c \, s_{TX}(t) + w_{RC}(t)
\end{equation}
with c the attenuation as the channel is supposed to be distortion free.

With the PAM, the error probability, $\Pr[\hat{a_k} \neq a_k]$ is
\begin{equation}
  P_e = 2 \left(1-\frac{1}{M}\right)\, Q\left(\sqrt{\frac{6}{M^2-1}\cdot \frac{E_s}{N_{_0}}}\right)
\end{equation}
with M the cardinality of the modulation and $N_{_0}$ the PSD of the noise.

The probability to receive a wrong symbol depends on
\begin{enumerate}
  \item the cardinality of the modulation scheme
  \item $E_s$ the average signal energy
  \item $N_{_0}$ the PSD of the noise
\end{enumerate}

\subsubsection{Filtering}
\begin{figure} \centering
  % \documentclass{standalone}
% \usepackage{tikz}
% \usetikzlibrary{automata,arrows}
%
% \begin{document}
  \begin{tikzpicture}
    \node[blk] at (-2,0) (b) {LTI};

    \draw[-stealth] (-4,0) -- node[above]{$x(t)$} (b);
    \draw[-stealth] (b) -- node[above] {$y(t)$} (0,0);
  \end{tikzpicture}
% \end{document}

  \caption{}
  \label{}
\end{figure}
\paragraph{Time domain}
In the time domain we find that the principal property of a filter are:
\begin{esp}
  m_y &= m_x \, \int_{-\infty}^{+\infty} g(t dt) \\
  r_{yx}(\tau)&=(g * r_x) (\tau) \\
  r_y(\tau) &=(g_{-}^{*} * r_{yx}) (\tau) \stackrel{(1)}{=} \left[\left(g_{-}^{*} * g \right)* r_x\right] (\tau)
\end{esp}
where $g_{-}^{*}(t) = g^{*}(-t)$

\paragraph{Frequency domain}
On the frequency domain the results are
\begin{esp}
  m_y &= m_x \, G(0) \\
  \P_{yx}(f) &= G(f) \P_x(f) \\
  P_{xy}(f) &= G^{*}(f) \P_{x}(f)\\
  P_y(f) &= |G(f)|^2 P_x(f)
\end{esp}

\begin{figure} \centering
  % \documentclass{standalone}
% \usepackage{tikz}
% \usetikzlibrary{automata,arrows}
% \tikzset{blk/.style={draw, minimum size=0.5cm, text width = 1.5 cm}}
% \tikzset{chan/.style={draw, minimum size=0.5cm, text width = 1.5 cm}}
% \begin{document}
% $\mathcal{P}$ vs $\math{P}$
  \begin{tikzpicture}

    \node[blk] at (-4,-2) (sync) {Synchroni-zator};
    \node[blk] at (0,-2) (chanE) {Channel extimator};
    \node[blk] at (0,-3) (timeE) {Time extimator};
    \node[blk] at (0,-4) (freqE) {Frequency extimator};
    \node[blk] at (2,0) (sig) {Signal Projector};
    \node[blk] at (5,0) (q) {Detector};
    \node at (-7,-6.5) (bla) {} ;
    \node at (-7,1) (1bla) {} ;

    % \draw[-stealth] (-4.5,0) -- node[above]{$r(t)$} (q);
    \draw[-stealth] (-7,0) -- node[above left]{$r(t)$} (sig);
    \draw[-stealth] (q) -- node[above]{$\hat{a_k}$}  (7,0);
    \draw[-stealth] (sync) -- (-2.5,-2);
    \draw[-stealth] (-6,0) -- (-6,-2) -- (sync);
    \draw[-stealth] (-2,0) -- (-2,-2) -- (chanE);
    \draw[-stealth] (-2,0) -- (-2,-3) -- (timeE);
    \draw[-stealth] (-2,0) -- (-2,-4) -- (freqE);
    \draw[-stealth] (sig) -- node[above] {$\hat{r_k}$} (q);
    \draw[-stealth] (chanE) -- (2,-2) -- (sig);
    \draw[-stealth] (chanE) -- node[above]{$t_{_0}$} (2.3,-2) -- (2.3,-0.6);
    \draw[-stealth] (chanE) -- (2.3,-2) -- (2.3,-5) -- (-4,-5);
    \draw[-stealth] (timeE) -- (2.3,-3) -- (2.3,-4.7) -- (-4,-4.7);
    \draw[dashed, red] (-2.6,1) rectangle (6.7,-6);
    % \draw[-stealth] (q) -- node[above]{$\hat{a_k}$} (4,0);
  \end{tikzpicture}
% \end{document}

  \caption{The realistic receiver scheme. The synchronization block can be based on packet communication or continuous transmission}
  \label{}
\end{figure}

$a_k$ and $h_{TX}$ depends only on the modulation.

\begin{figure} \centering
  % \documentclass{standalone}
% \usepackage{tikz}
% \usetikzlibrary{automata,arrows}

% \begin{document}
  \begin{tikzpicture}
    \node[blk] at (-2,0) (b) {$h_{RC}$};
    \node[blk] at (2,0) (q) {Detector};

    \draw[-stealth] (-4.5,0) -- node[above]{$r(t)$} (b);
    \draw[-stealth] (b) -- node[above] {$t_0 + kT$} (q);
    \draw[-stealth] (q) -- node[above]{$\hat{a_k}$} (4,0);
  \end{tikzpicture}
% \end{document}

  \caption{}
  \label{}
\end{figure}
\subsubsection{Advanced modulation techniques}
The channel usually doesn't have a flat response over all the frequencies, so we need to differenciate the different types of channels. We call \textbf{ frequency selective channel} a channel that varies fastly (w.r.t. the center frequency), so that we use only small bands to consider the subchannel a flat one. We call, instead, a channel \textbf{Wideband} if it can be approximated
to a flat one for a great frequency band (w.r.t. the center frequency). A wideband channel has some other problems different from the frequency selective channel: it's more subject to ISI, Inter Symbol Interference. For theese type of channels we need suitable Tx and RX filters and we need to equalize the channel, which means that the receiver need to compensate distortion.

\section{Other modulations}
PAM, QAM and PSK, for example, are called single carrier modulations as they depend only on one center frequency $f_0$. We could use, instead, other types of modulations:
\begin{enumerate}
  \item Multi carrier modulations

\begin{enumerate}
  \item OFDM, the Orthogonal Frequency Division Multiplexing
  \item Vector coding
\end{enumerate}
\item Spread spectrum
\end{enumerate}

The first is used for high bitrate application, wherease the second is more suitable for low rate/low power application

\subsection{Multi carrier modulation}
The main concept is to partition the frequency band in smaller ones with bandwith $B_{_0}=\frac{1}{T_{_0}}$. We will have then many small subchannels with bandwith $B_{_0}$ and centered on a particular frequency: $B_i \in \left(f_i - \frac{B_{_0}}{2};f_i + \frac{B_{_0}}{2}\right)$.

\begin{figure} \centering
  % \documentclass{standalone}
% \usepackage{tikz}
% \usetikzlibrary{automata,arrows}
%
% \begin{document}
  \begin{tikzpicture}
    \node[draw] at (-3,0) (mod) {QAM modulator};
    \node[draw] at (0,0) (sp) {S/P}; %rectangle

    \node[draw] at (4,4) (q1) {digital demodulator};
    \node[draw] at (4,2) (q2) {digital demodulator};
    \node       at (4,0) (qi) {$\vdots$};
    \node[draw] at (4,-2) (qn) {digital demodulator};
    % \node[draw, rectangle] at (4,4)  (4,-2) (ibmap) {IBMAP};
    \node[draw] at (7,0) (ps) {P/S};

    \draw[-stealth] (-5,0) -- node[above]{$b_l$} (mod);
    \draw[-stealth] (2,4) -- node[below]{$T$} (q1);
    \draw[-stealth] (2,2) -- node[below]{$T$} (q2);
    \draw[-stealth] (2,0) -- node[below]{$T$} (3,0);
    \draw[-stealth] (2,-2) -- node[below]{$T$} (qn);

    \draw[-stealth] (q1) -- node[below]{$T$} (7,4);
    \draw[-stealth] (q2) -- node[below]{$T$} (7,2);
    \draw[-stealth] (5.5,0) -- node[below]{$T$} (6.6,0);
    \draw[-stealth] (qn) -- node[below]{$T$} (7,-2);
    % \draw[-stealth] (ibmap) -- node[above]{$c(n T_s)$} (ps);
    \draw[-stealth] (ps) -- node[above]{b} (9,0);
  \end{tikzpicture}
% \end{document}

  \caption{}
  \label{}
\end{figure}


$T_{_0}=N\,T \gg T$

Given that each subchannel has different gain, the receiver will experience diferent SNR ar each subchannel. OFDM uses different solutions to overcome this:
\begin{itemize}
  \item Power loading, where the transdmitter allocates more power over the best subchannel
  \item Bit loading where the transmitter sends more bits over the best subchannel
\end{itemize}

\subsubsection{Spread Spectrum}
This technique is used in GPS, GALILEO,UMTS and Bluetooth, for example. The idea is to use more bandwidth than the one required to transmit a digital stream with symbol period T: $B \gg B_{min}=\frac{1}{T}$.

There are some pros and cons to the usage of Spread Spectrum. The main pros are
\begin{itemize}
  \item Low probability to intercept the signal
  \item Resistance to a jamming signal
  \item Resistance to ISI
  \item Multiple access
  \item Better localization abilities
\end{itemize}
We can cite three main types of spread spectrum: the \textbf{direct sequence}, the \textbf{time hopping} and the \textbf{frequency hopping}

\section{Transmission media}
We now consider two main transmission media: the transmission lines and cables and the optical fibers.

The transmission lines have a frequency response that can be written as
\begin{equation}
  G_{ch}(f) = e^{-(1+\jmath)\sqrt{\frac{f}{f_c}} - \jmath 2 \pi f t_c}
\end{equation}
where
\begin{itemize}
  \item $f_c = \left(\frac{1}{kd}\right)^2$  ??? aggiungere qualcosa forse
  \item $t_c = d \sqrt{l c}$ is the propagation delay and depends on the distance (d), the cable capacitance (c) and the cable inductance (l).
\end{itemize}

The attenuation can be rewritten as
\begin{esp}
  (a_{ch}(f))_{dB} = (\tilde{a_{ch}(f)})_{dB/km} \, d_{km}
\end{esp}
and the specific line attenuation can be rewritten as
\begin{equation}
  (\tilde{a_{ch}(f)})_{dB/km} \approx 8,68 \alpha(f) = a_{ch}(f_1) \sqrt{\frac{f}{f_1}}
\end{equation}

Two things can be remarked on transmission over cables: if twisted pairs are used, there's a problem  on the cross talk, which means that the electrical signal in one cable induces currents on the nearby cables producing an interference. We define the cross talk interference as
\begin{description}
  \item[NEXT] Near end cross talk, if the transmitter and receiver are at the same side
  \item[FEXT] far end cross talk, if the transmitter and receiver are at the opposite side
\end{description}
\subsubsection{Optical fibers}
The channel gain for the optical fibers can be written as
\begin{equation}
  G_{ch}(f) = A_F^{-1} e^{-2(\pi f \sigma_F)^2 - \jmath 2 \pi f t_F}
\end{equation}
where
\begin{itemize}
  \item $A_f$ is the attenuation coefficient
  \item $\sigma_F$ is the dispersion coefficient and depends on the material, the geometry of the fiber and linearly on the distance
  \item $t_F = \frac{d \, n}{c}$ is the propagation delay and depends on the distance and the light speed on the fiber material.
\end{itemize}
In general the bandwidth carried by the optical fiber can be modeled as $B_{FIB} \propto \frac{1}{\sigma_{F}}$


\section{Characterization channels}

\textbf{Insert the block figure}

We can make an equivalent model of the previous one by representing it as an electrical one.
We can model each block as bipols or double ones as it can be seen in the figure below.


\subsection{Electrical Model}
\textbf{Insert the electrical model}

In this model we can say that the system has a \textit{noise temperature} ($T_s$) and the receiver RC contribute with a temperature and a figure noise, rispectively $T_a$ and $F_a$, the last one of which is $F_a = 1 + \frac{T_a}{T_0}$.

Using this new approach we can describe the model counting on the following hypothesis:
\begin{enumerate}
\item The \textit{thermal noise} is AWGN.
\item All noise sources are in the channel.
\item We consider the noise contributions at the channel output, coinciding with the receiver input.
\end{enumerate}

With these assumptions we can describe the \textit{Effective Temperature} $T_{eff,RC}$ such that:
\begin{equation}
T_{eff,RC} = T_s + T_a
\end{equation}
with $T_s = T_0 = 290K$

We want to quantify the noise power by the \textit{Power Density of the noise} $P_{w,RC}(f)$ such that
$$P_{w,RC} (f) = \frac{k_b \ T_{eff,RC}}{2}$$
With $\mathcal{P}_{w,RC}$ the \textit{Power Spectral Density} that we can assume constant
\begin{equation}
\mathcal{P}_{w,RC} = P_{w,RC}\ \frac{\mid Z_{RC,in}(f)\mid ^2}{R_{RC,in}(f)} = \frac{N_0}{2}
\end{equation}

We assume we have a narrowband channel where the frequency response of the channel is almost constant over the frequency band $\mathcal{B}$ of the signal.

We call C the gain of the channel such that
\begin{equation}
C = \mid \mathcal{G}_{CH}(f_0) \mid \ \forall f \in \mathcal{B}
\end{equation}

It's also important to remember that the channel induce some delay as it can be seen in the expression
\begin{equation}
t_0 = -\frac{1}{2pif_0} \phase{\mathcal{G}_{CH}(f_0)}
\end{equation}
Where $f_0$ is the carrier frequency so that $s_{RC}(t) = C\ s_{TX}(t - t_0)$.

\paragraph{Signal to Noise Ratio (SNR)}
It can be considered two different approaches to infer this kind of parameter:
\begin{enumerate}
\item Statistical SNR $\Gamma_{Stat}$
\item Electrical SNR $\Gamma_{Elec}$
\end{enumerate}

In case of (A) ergodicity of the signals and (B) matching between input and output impedences we can say that the two previous SNRs are equal.
\begin{equation}
\Gamma_{Stat} \doteq \Gamma_{Elec}
\end{equation}

\subparagraph{Statistical SNR}
We are interested in the statistical useful components of our received signal, as we see $r(t) = s_{RC} (t) + w_{RC} (t)$.

We so define its \text{Statistical Power} $M_{s_{RC}}$ such that
\begin{equation}
M_{s_{RC}} = C^2 \ M_{s_{TX}}
\end{equation}

If we are interested on the noise component we can describe its \textit{Statistical Power} as:
\begin{equation}
M_{w_{RC}} = \int_\mathcal{B} \mathcal{P}_{w_{RC}}(f) df = \frac{N_0}{2}\ 2B = N_0\ B
\end{equation}

So that
\begin{equation}
\text{Statistical SNR:} \Gamma = \frac{M_{s_{RC}}}{M_{w_{RC}}} = \frac{M_{s_{TX}}\ C^2}{N_0\ B}
\end{equation}


\subparagraph{Electrical SNR}
Given the transmission power $P_{TX}$ and the attenuation $a_{CH}$ of a channel which introduces some kind of noise, we can firmly say that this last parameter is related with the channel's gain such that:
\begin{equation}
a_{ch} = \frac{1}{g_{ch}}
\end{equation}
Now it can be discovered the received power $P_{RC}$ as:
\begin{esp}
P_{RC} &= \frac{P_{TX}}{a_{CH}} \text{, so that}\\
P_{w,RC} (f) &= \frac{k_B \ T_{eff,RC} (f)}{2}
\end{esp}

And the electrical SNR is:
\begin{esp}
\Gamma_{Elec} &= \frac{P_{RC}}{P_{w,RC}\ 2B} \\&= \frac{P_{TX}}{k_B \ T_{eff,RC}\ B a_{CH}}
\end{esp}


In the logarithmical scale we can say that the electrical SNR is equal to the following \textit{Link Budget Expression}
\begin{esp}
(\Gamma_{Elec})_{dB} = 10\log \Gamma_{Elec} \\& = (P_{TX})_{dB} - (a_{CH})_{dB} - 10\log (10^9 \ k_B T_0) - 10\log (\frac{T_{eff,RC}}{T_0}) - 10\log (B)_{MHz}
\end{esp}

And we can say that $T_s = T_0$ and $10 \log (\frac{T_{eff,RC}}{T_0}) = (F_A)_{dB}$

\subsection{Full communication Link}
Add the figure of the communication link.

\begin{equation}
\text{Overall System Error: } \Gamma_{PCM} = \frac{E[\mid a(t)\mid ^2]}{E[\mid \widetilde{a}(t) - a(t)\mid ^2]}
\end{equation}

\subsection{Errors}
Two type of errors:
\begin{enumerate}
\item Quantization errors $e_q$
\item Detection errors $e_{B_c}$ referring to the offset between $\hat{b_l}$ and $b_l$ such that $\hat{b_l} \neq b_l$
\end{enumerate}

We know that $e_q$ and $e_{B_c}$ are uncorrelated so that they sum up to the expression
\begin{equation}
e(nT_s) = e_q(nT_s) + e_{B_c}(nT_s)
\end{equation}

Using these expressions we can say that:
\begin{esp}
\text{Overall System Error: } \Gamma_{PCM} &= \frac{E[\mid a(t)\mid ^2]}{E[\mid \widetilde{a}(t) - a(t)\mid ^2]} \\&= \frac{M_q}{M_{e_q} + M_{e_{B_c}}}
\end{esp}

\textbf{Insert the expression A)}

\subsection{Spectral Efficiency}
The Spectral Efficiency is a quantity $\nu = \frac{R_b}{2B_{min}}$ that describe how many bits per time are sent with the conventional $B_{min}$ and dimensionally is [bit/s/Hz], referring to the rate due to a specific band.

This quantity has a bound (\textit{Shannon Bound}) which we cna express as:
\begin{equation}
\nu \leq \frac{\log (1 + \Gamma)}{2}
\end{equation}
with $\Gamma = \frac{E_s}{N_0}$ which is the reference SNR.

\textbf{Insert the graphic}

In Telemedicine and Biomedical Applications we use PSK, QAM and PAM modulations quite often, but also ORTHOGONAL and Bi-ORTHOGONAL. These last ones seem not so efficient but can reach the very low probability of error for lower SNR using lower power to get the same SNR and reach the same $P_{bit} = 10^-6$).

\section{Radio Links}
Every time we want to convey informations through electricmagnetic field in the free space we have a full spectrum of frequencies and we know there are possibly systems working at some frequencies in this spectrum like those who work on Very LOw Frequency - VLF(f = 3Hz), Low Frequency - LF, Medium Frequency - MF, High Frequency - HF, Very High Frequency - VHF, Super High Frequency - SHF, Extra High Frequency - EHF, microWAVES(f = 300GHz, $\lambda$ = mm).
Hence it's safe to say that the choice of the carrier frequency $f_0$ can be done upon different requirements like some features of the antennas.

Also using radio links we have to be aware of the \textit{propagation effects} like reflections and scattering that bring our signal not to follow just the \textit{Line of Sight path} (LOS) but might differ from this one. In fact r(t) can also follow other paths so that the receiver can receive LOS plus other components deriving from other structures that might attenuate the original signal (i.e buildings). In this case the signal is called \textit{multipath} and in some case it could be also an advantage scenario.

We could use the Maxwell equations but we prefer to use a simplified model:
\paragraph{Narrowband Model Channel}
We're referring to a scenario where there are two antennas, a trasmitting and a receiving one ($Ant,Tx$ and $Ant,RC$).

For the transmitter point of view (TX) we can have the power density for a isotropic antenna differing from a directional one such that:
\begin{equation}
\begin{cases}
\Phi_0 = \frac{P_{TX}}{4pi(d^2)} \ \text{ \textit{for an isotropic antenna}}\\
\Phi = \Phi_0 \ g_{Ant,Tx} \ \text{ \textit{for a directional antenna}}
\end{cases}
\end{equation}

We see that the second one PSD is much higher than the previous one.

For the receiver we have
\begin{equation}
P_{RC} = ? \ A_{Ant,RC} \eta_{Ant,RC}
\end{equation}
with $A_{Ant,RC}$ referring to the effective area of the receiving antenna and $\eta_{Ant,RC}$ the efficiency factor.

In this case we have:
\begin{equation}
\begin{cases}
\text{Power Density: } P_{RC} &= P_{TX}\ \frac{g_{Ant,TX}}{4pi(d^2)}\ A_{Ant,RC}\ \eta_{Ant,RC}\\
\text{Antenna's Gain: } g_{Ant} &= \frac{4pi}{\gamma^2} A_{Ant,RC} \ \eta_{Ant,RC}
\end{cases}
\end{equation}

We need a smaller antenna to have the same gain $g_{CH}$ at a higher carrier frequency $f_0$:
\begin{equation}
\begin{cases}
\text{Frequency Response: } \mathcal{G}_{CH} (f) = \sqrt{g_{ch}} e^{-j2pif{t_0}}\\
\text{Channel's Attenuation: } a_{CH} = \frac{4pi(d/\gamma)^2}{g_{Ant,TX}\ g_{Ant,RC}}
\end{cases}
\end{equation}
With $t_0$ the propagation delay

Certainly $a_{CH}$ can be written in the decibel scale as:
\textbf{Copia descrizione in decibel da pitt}

In conclusion we can highlight that using radio links we have less attenuation when we increase the distance and an advantage also in the carrier frequency but we have to face the fact that we need a much larger bandwidth since we are not conveying electricalmagnetic waves into a medium.

\subsubsection{Propagation Effects }

\begin{enumerate}
\item Path Loss $\Rightarrow$ attenuation
\begin{itemize}
\item Distance TX - RC (Typical values: 100 - 1000 m)
\item Propagation effects of the channel
\end{itemize}
\item Shadowing $\Rightarrow$ attenuation
\begin{itemize}
\item Absorption
\item Reflection
\item Scattering
\end{itemize}

As we previously said, if ther are obstacles in the space of free-moving of the signal it can be scomposed. The typical distance are the lenghts of the obstacles and this express a high dependence from the enviroment. In fact if we refer to the outdoor environment the distance (d) from TX and RC is between 10 to 100 m, whereas for the indoor one d $\leq$ 10 m.

$$\text{Path loss and shadowing = large scale propagation effects}$$
\item Multipath fading: constructive and distructive addition of the multipath signal components (rapid phase changes).
Typical distances are very very short like in the range of wavelength of the wave travelling and considered.
\end{enumerate}

\textbf{Inserisci grafico 2}
Linear model if we consider just the path loss fenomenum but if we consider both the path loss and the shadowing we have a model with a variable characteristic. If we add also the multipath factor the function $(\frac{P_{RC}}{P_{TX}})_{dB}$ will have even more flactuations

\chapter{Electroencephalography}
The EEG signal comes from the most exterior part of the brain and it's an electrical signal. The outer part of the brain contains the synapsis, wich increases whith aging of the person.
[slide 3]
The neural network can be tree-based and if all neurons in a tree are activated at the same time, we get the EEG signal.

[Slide 4]
The EEG propagation can be modeled as
\begin{equation}
  \underbrace{EEG}_{P_rc}= \underbrace{LPF}_{P_{TX}} + \underbrace{propEffects}_{shadow/fading} + \underbrace{artefacts}_{noise}
\end{equation}

In fact, putting an electrode in a point we get the signal coming in the "line of sight" and the propagation of other signals which are transmitted via the skull and the scalp, but aren't in the LOS.

\section{Brain activity characterization}\label{sec:bac}
In order to study the EEG signal, we need to take into account the activity of the brain in the \begin{itemize}
  \item Time domain
  \item Frequency domain
  \item Space domain (the topology)
\end{itemize}
\subsection{Time domain}
[slide omonima]
In the time domain we have that the power in the movement situation (figure bottom left) of the signal is that it starts to become constant, wherease in the resting state, there are three main lobes.

\subsection{Frequency domain}
The frequency domain of the EEG goes in the interval $[0.01;500] Hz$, but EEG machines, usually, don't go under 1.4 Hz and above 60 Hz
To receive other frequency we need to use other techniques.
[slide rhythms]
The activity of the brain can be divided into different types of frequencies. The amplitude is reversly proportional to the frequency, so with high frequency the signal is almost distrupted by path loss and shadowing.

\subsection{Space domain}
We know that some region are used by the brain for specific activities. Not completely true, butgeneral tendency. In the slides there are the somatosensoty strip and motor strip. In the motor strip, if we move the hand, it's easier to get signals (most reactive region of the brain) in the brain when hand movement is done.

The homunculus plot we have some part of the body given which area is excited when something is done. The today area is not to use the functional approach, but the connectivity/circuit approach: brain seems to be more likely to have circuits to enable some activities. To accomplish some functions we have some circuits activating together to accomplish that function.

\section{Acquisition of EEG}
We revidse the comparison between an EEG signal and a wireless transmitted signal:
\begin{esp}\label{eq:eeg-wireless}
  EEG &= LPF + \text{propagation effects} + \text{artifacts}\\
  P_{RC} &= P_{TX} + \text{path loss/shadowing} + \text{noise}\\
\end{esp}
where LFP Local Field Potential

\subsection{The inverse problem}
The electrode is a signal source which sends some test signal $s_{TX} = \delta(t)$ to test the received signal into another electrode. The path loss is proportional to $e^{-d}$ and introduces a delay $\tau_0$. We know that there's a kind of pathloss as in the model shown in \eqref{eq:eeg-wireless}

If we set more than one source transmitting a known signal (e.g. $\delta(t)$), we can collect all the delay thanks to the sensors and we can get a matrix which describes the propagation effects (delay, attenuation, cross attenuation) from each source to the sensors.

Inverting that matrix leads to the characterization of the brain. The problem is that we don't actually have a known transmittable signal without other ones, so we need to rely on models known. A reliable source model is an open problem.

We need to characterize the signal as said in \autoref{sec:bac}. We have brain (system) with the neurons. The communication of neurons happens in groups of neuron. Each group can be part of the same circuit (motor, perception/touch, sight, acoustic). There are some circuits interacting with each other creating the system. The visual circuits and the motor one are well studied and well known; other are less known. Now there are studies on why and how the brain ``stop'' of some circuits if other are working. For example Moser studied neurons that activates while the individual is handling space navigation (movements). Same set of neurons works only in some specific frequencies. Neurons can code signals in time, frequency and in phase.

\subsubsection{Sensori-motor circuit}
The cortex and basal ganglia communicates with some circuit. There are some interesting **
for example, if we want to move the arm, the cortex (basal ganglia) process the tactile info of the hand (e.g. resting on a table) and uses this information, transmitted with a specific frequency, that will be received and processed to specify a movement. The reception of the signal is slow (f=6Hz) The updating of a movement is quite faster instead (f>20Hz).

As the received power is inversly proportional to the frequency, the brain need to put more energy to high-frequency signal, but brain is constantly in power-save mode, so uses the higher frequency to propagate the signal in a smaller space, wherease the lower frequency for long-distance signals.

The neurons need to be synchronized and this is a topic that's being studied. The synchronization is used to lower the transmission energy, as the two groups are ready to transmit/receive without interfering.

\subsubsection{Circuit modeling}
One useful and needed approach is the multi-modal: to see the same system we record data from different sensors (EEG, magnetic information, oxigen consumption).
Another option is stimulating the brain to test the circuit. It's not possible to send a signal and record it from the sensor. We though use some non-invasive techniques to approximate the test signal. An example is the TMS (transcranial magnetic stimulation); another example is the muscular stimulation: we let the individual do some actions and we activate some regions. We can record the signal that was produced.

\section{Recording}
When we record the EEG we need to set the montage of the electrods and the the cap, which sets where to put the sensors. Importants are two references: the INION (Iz) on the back of the head and the Nasion, the front of the brain. The electrods are made analyzing the left and right preauricular distance to set the electrodes with a specified proportion. To make it easier, there's a standard cap.

The signal is acquired using a differential amplifier, which makes the difference between the reference and the signal.

The standard montage has 10-20 systems which corresponds to 200 electrodes approximately. The sampling needs a frequency at least fo 256Hz with 12/24 bit to represent the levels. Some systems are wired, other are wireless. The portable version is appreciated as less intrusive.
\subsection{Paradigm}
To record the EEG signal we typically need to design the paradigm: typically we don't simply put the cap and record, but we use a paradigm to have a better signal. At the beginning of the session, the doctor tells the patient what he will do while the acquisition is on, the stimuli given to the patient. A different stimuli produces different response of the brain, so the recording can be falsified and almost-unreadable as the signal can be corrupted. So not knowing the state of the person can lead to an useless signal.
the acquisition goes as follows
 Typically the person does nothing when the screen is blank. when something else is shown, the person enters in a warning state, then it receives some other stimulus and does some actions for a fixed amount of time and then goes to rest. This sequence is called \textbf{trial}. In order to compensate variability of the brain, the trial is repeated.

 We can measure the resting state to analyze the brain activity related to the memory dealing with the environment stimulous. In case of stimuli, we can analyze the event-related potentials (event=stimulus) and event-related (de)synchronization of the signal.

 \subsection{Artifacts}
 When we record the EEG we have 2 types of artifacts:
 \begin{itemize}
   \item Powerline noise given by the coupling noise of the amplifier
   \item electrode displacement, given by the misplacing of an electrode, the touch of an electrode and the TMS, for example
 \end{itemize}
 Muscular activity (around the face): swallowing, eye closing, mouth movement/chewing. Each signal has a frequency which is specific (eye ~~ 3Hz, chewing ~~ 10Hz) and the signal is spatial specific. Alpha waves overlaps, for example, with eye movement, so the signal need to be cleaned from the eye artifacts.

 The pre-processing phase consists in clean the signal from artifacts, enhance SNR, some temporal filters and spatial filters, which exploits the info given by different sensors in different positions to enhance the SNR.

 Sometimes we use the interpolation, we surely segments the signal and it's useful to decompose the signal by separating the frequencies that compose it (PCA and ICA).

 We need to select which processing of the signal. Surely the pre processing explots filtering, segmentation and one of PCA or ICA.

 When we stimulate, there's a drop of the signal (fast, 1s) and it takes a lot to recover the signal (10s) that may be discarded. Some informations are transmitted in 0.1 s, so 10s is really a lot of wasted signal that can though be used if appropriate filtering is applied.

 One criticality is*

 which can lead to the non-stimulation of the patient and waste time.

 One solution is the zero-padding when the signal drops and filtering the really low component, to be able to recover the signal without wasting it.
 High frequency signal (teeth clanching - scritto male-) can be eliminated with LPF.

 One spatial filter to suppress noise is the laplacian. The small laplacian is used if electrodes filtered are near, Large laplacian if they're far away. To get a higher SNR a temperature sensor can be useful.

 We can use a low pass filter or high pass filter to attenuate some frequencies. Other useful filters are the notch, which eliminates a specific frequency, or a bandpass filter to consider only a band of frequencies.

 \subsection{Segmentation}
 The acquired signal is divided into smaller time signals if the signal is non-stationary, so to make it stationary. Typical length is 2s (deterministic). On the segment, we can decide automatically or by-eye to discard a signal if it's too bad. On each segmen we can compute some statistical properties such as the amplitude distributions (standard deviation, kurtosis, skewness). For example, we can discard a signal that has a standard deviation higher than a threshold as it's considered too noisy. To eliminate trials with some artifacts we can set a threshold that, if exceeded, set the signal as ``bad''

 \subsection{Independent Component Analysis (ICA)}
 Different signals which spatial derivation is known, can be mapped into another signal space to map it to another signal space. In this new signal space, ICA can identify artifacts and delete them (e.g. artifacts related to muscular movements)

\chapter{Spread Spectrum}
30/10/2017 lect \# 10
Spread spectrum is an advanced modulation technique (like ODFM) suited for low power sensors (important in telemedicine).\\
Basic characteristic: the transmitted signal bandwidth is much larger than the minimun required one ($B >> B_{min}$). If we have larger bandwidth we have low probabibilty of intercept/jamming, much more robustness against ISI, multiple acces, localizaition capabilities. \\
3 modulation schemes:
\begin{itemize}
  \item Direct sequence (DS)
  \item Frequency - hopping (FH)
  \item Time - hopping (TH)
\end{itemize}
(Applications: GPS-UMTS-Galileo-...)
\section{Principles}
\begin{itemize}
  \item Symbol period $T$
  \item $\log_2 \frac{M}{T}$ bits per second
  \item received signal r(t)
\end{itemize}
wrt to the classic modulation scheme: add some operations after digital modulator in order to spread the signal and some operations before digital demodulator in order to recover the original signal.\\
We need a spreading code $s_C(t)$ independent of the data. Spread tx signal: $s_{TX}(t): B_S = B_C + B$.\\
To perform the \textbf{spreading}: multiply the information signal $s(t)$ with the spreading signal $s_C(t)$. Resulting signal has much larger bandwidth than the original one.\\
At the receiver: \textbf{despreading}. We need the knowledge of the spreading code and of the timing. Correlating $r(t)$ with a synchronized copy of $s_C(t)$, m-branch structure.
Robustness wrt ISI $\rightarrow$ $E_J$ total energy of $I(t)$ (waveform), $s_{IJ}$ randomly chosen (interference), received signal $r(t) = s_i(t) + I(t)$.\\
Output of i-th correlator: $r_i(t)\rightarrow SINR = \frac{E_s}{E_j}\frac{N}{M}$. The interference is spread on a larger space, the signal can be easily retreived. SIR increased

\subsection{Direct Sequence scheme}
chip rate: $\frac{1}{T_c} \sim B_c$. $B_c$ about $B_c/B \sim T/T_c$ times larger than $B$\\
frequancy domain $\rightarrow$ convolution between $S(f)$ and $S_c(f)$.\\
\textbf{direct sequence modulator/demodulator image}\\
$\hat{x}(t)$ received signal, then multiplied by a synchronized version of the spreading code. \\
spreading/despreading calculations: slides.\\
recovered signal $\hat{s}(t)$ made of two terms: $s_l$ (information, the symbol we tx) + $n_l$ (noise) $\rightarrow$ can be demodulated with a standard modulator, if we just have noise (2 assumptions: no multipath, no ?, AWGN channel) we dont't need anything different than a standard demodulator even if we spread the signal (no impact of spreading on AWGN channels). \\
received signal $r(t) = s(t)s_C^2(t) + \underbrace{n(t)s_C(t)}_{\textrm{approx same distrib as } n(t)}= s(t) + n'(t)$.\\
In case of Narrowband interference I(t) (figures), at the receiver: spread signal + additive narrowband interference. Through despreading: despread the useful signal, spread the interference (multiply again by a copy of the spreading code) $\rightarrow$ the demodulator acts as a low pass filter and removes most of the energy of the spread interference.\\
Multipath and ISI: at the rx convolution of the received signal with the impulse response of the channel $\rightarrow$ delayed version of the signal AND of the spreading code. Despreading using a version of $s_C$ synchronized with the first component of the multipath. Second multipath componend remains spread, we can efficiently recover the first multipath component, the spread spectrum is also efficient in case of multipath.\\

\section{Exercises}
\begin{exercize}{3}
  Audio signal piecewise with duration of $10$ ms. 4 possible statistical distributions:
          \begin{itemize}
            \item[1)] $\mathcal{N}(0, \sigma_1^2)$, $\sigma_1 = 1$ V
            \item[2)] $\mathcal{N}(0, \sigma_2^2)$, $\sigma_2 = 2$ V
            \item[3)] $\mathcal{U}([-5, 5])$
            \item[4)] $\mathcal{U}([-12, 12])$
          \end{itemize}
          Sampling frequency $F_S = 64$ kHz. Quantization: bank of parallel quantizers (only 1 active at time).  Bit stream in output $\rightarrow$ pkts with vatiable length. Requirements:
          \begin{itemize}
            \item[a)] $\tilde{R}_b \downarrow$, $\Lambda_q \geq 25$ dB.\\
            \begin{equation}
                P_{sat} =
            \begin{cases}
              \leq 63 \cdot 10^{-5} & \textrm{types } 1, 2 \\
              0  & \textrm{types } 3, 4
            \end{cases}
          \end{equation}
          \begin{solution}{a}
            For gaussian distributed signals: $$P_{sat} = 2\mathcal{Q}(\frac{v_{sat}}{\sigma_{sat}} = 6.3 \cdot 10^{-5}) \rightarrow v_{sat} = 4\sigma_a$$
            $$\Lambda_q = \frac{M_a}{M_{eq}} \simeq \frac{\sigma_a^2}{\Delta^2/12} = \frac{12\sigma_a^2L^2}{4v_{sat}^2} \rightarrow L$$
            $$b_{1,2} = 6 bit$$
            Uniform distr signals:
            $$v_{sat} = a_{max} = 5 V \textrm{(type 3)}, 12 V \textrm{(type 4)}$$
            $$(\Lambda_q)dB \simeq 6b_{3,4} 5 bit$$
          \end{solution}

             Assuming 60s recording.\\
            On average: 15s of type 1, 5s of type 3, 20s of type 3 and 4. $\tilde{R}_b$? (FIGURE)
          \begin{solution}{b}
            $R_b = bF_s$, $N = F_s T_{rec}$, $\tilde{R}_b = F_s\left[\frac{\textrm{rec time 1 and 2}}{60s}b_{1,2}+\frac{\textrm{rec time 3 and 4}}{60s}b_{3,4}\right] = 341$ kbit/s
          \end{solution}

               Assume that each piecewise has some prob of occurrence. Packet format for each piecewise type?\\
            \begin{solution}{c}
              Identify the quantiz: 4 quant $\rightarrow$ 2 bit\\
              Types 1 and 2: $6 \textrm{ bit} \cdot 64 kHz = 10 \cdot 10^{-3} = 3840$ bit\\
              Types 3 and 4: $5 \textrm{ bit} \cdot 64 kHz = 10 \cdot 10^{-3} = 3200$ bit\\
            \end{solution}
          \end{itemize}
\end{exercize}

----
Lecture 2017-10-31

Before autocorrelation: all is on slides, nothing added.

We define the autocorrelation of the spreading code

\begin{equation}
  \rho_c (t,\tau_1) = \frac{1}{T}\int_{0}^{T} s_c(t) ~ s_c(t-\tau_1) dt
\end{equation}
which, if $s_c(t)$ is periodic, then the autocorrelation itself is periodic with same period. Generally, if we compute the autocorrelation with another period, we find that the integral is equal to $\rho_c(\tau_{_0}-\tau_1)$

The autocorrelation function, as $s_c(t)$ is not an infinite-time signal, it's not like a $\delta$ function, so there are some studies that tries to approximate the autocorrelation with a $\delta$ function.

synchronization happens in coarse way: we define a power threshold and once a received signal has power higher than the threshold, we define it as the synchronization. this type is made in a discrete way (typically Tc). The other synchronization search for peaks in the autocorrelation func.

RAKE RECV

we synchronize to each stream $s_c(t) ~ s_c(t-\tau)~ s_c(t-2\tau)$ and then demodulate for each synchronized signal. Then we need to combine the output with a diversity combiner. For example with the MRC (MRC, Maximum Ratio Combiner) where the signal received with most power is used to extract the data transmitted by the signal itself. As there are different delays (all multiples of the chip time), we need to take into account the different phase shift.

Once we selected one component, the other are considered as interference.%and will be spread when decoding. not sure

\subsection{Frequency hopping}
The modulated signal is transmitted changing the carrier frequency every chip time $T_c$. At each time we select the frequency where to transmit based on the spreading code. If we use fast frequency hopping we have frequency diversity as we protect each symbol from narrowband interference. The digital modulator needs to have as input both the symbol and the cosine form, which must be given according to the spreading sequence.

At the receiver we have the opposite function, minding that the spreading sequence must be synchronized.

Performance almost all in slides
In FFH we change frequency multiple time in the same chip period, so the interference is reduced to $\frac{1}{N}$, N = number of changes. Wherease in Slow FH (SFH) we need to avoid errors due to interference of that subchannel.

with multipath, as the carrier frequencies are different and non overlapping, the signal that may interfere, is eliminated with the filters of the other band.
with FFH, the channel can be considered as flat fading, as there's no multipath interference, wherease in SFH the channel may be frequency selective.

\subsection{Time Hopping}
all on slides

Cross correlation between users is present

\chapter{Multi user systems, part III}
\section{Superposition coding}
In case of superposition coding, user 1, that has a better SNR, can correctly decode his message subtracting user 2's message from the received message. User 2 instead, that has a worse SNR, sees message intended for user 1 as additional noise. We can also extend this technique for more than two users. \\
Capacity with code division: we use non-orthogonal codes (Walsh-Hadamard is the most used) $\rightarrow$ formula on the slides. If we use non orthogonal codes, G is processing gain, interference from the other users treated as noise (reduction by 1/G). \\
In conclusion, we can say that superposition coding with successive interference cancellation overcomes orthogonal strategies.
\subsection{Common data}
Broadcast application. User with better channel gain always receives also data intended for user with worse channel gain, common data must be tx to both users.\\
\subsection{Downlink capacity}
Real case scenario: no AWGN channel, we need to account for fading, we can assume that users have indipendent channel gains changing over the time, they have CSI at tx and rc and slow fading. $\rightarrow$ Ergodic capacity/ Zero-outage capacity/Outage capacity/Minimum rate capacity (slides).
\subsection{Uplink capacity}
AWGN channel, K tx, one receiver, single antenna. Two- user MAC cpacity: two costraits $$R_k \leq B\log_2(1+\frac{g_k P_k}{N_0 B})$$ $$R_1 +R_2 \leq B\log_2(1 + \frac{g_1P_1 + g_2P_2}{N_0 B})$$ Can be extended for $K > 2$.\\
It is also possible to develope a theory for uplink capacity in fading $\rightarrow$ independent users with independent channel gains that can change over time, single antenna both at tx and rc, CSI at tx and rc. Ergodic, Zero-outage, Outage, Min rate capacity.
\section{Power control}
Another important issue when designing a network. Goal: adjust the tx power of the users, depending on the SINR of each user in order to meet a given threshold. Different methods for uplink (more straightforward, only 1 tx)/ downlink (more challenging).
Uplink case (k interfering users), SINR: $$\gamma_k = \frac{g_kP_k}{n + \rho \sum_{j \neq k} g_jP_j}$$
Assumption: each link requires a min SINR greater than zero. \\
$P$: vector of tx powers, $u$: vector of noise powers, F matrix (slides). $P^*$, pareto optimal solution. \\
Iterative power control algorithm: Foschini-Miljanic. $P(i+1) = FP(i) + u$ \\
Feasible region: all the power vectors that can achieve the target SNR. $P^*$, minimum power vector in this region.
Uplink-downlik duality: we can extrac info from one channel for the other. K user downlink and uplink dual if:
\begin{itemize}
  \item same channel responses $h_k$
  \item each channel in downlink and uplink has the same noise statistics
  \item ?
\end{itemize}
Multiuser diversity: opportunistic scheduling $\rightarrow$ in the wireless context results to be not very fair and long delays\\
Possible solution: proportional fair scheduling, not as high throughput as opportunistic scheduling, but more fair and reduces latency.
\section{Multiple antennas}
slides

\chapter{Ad-Hoc Wireless Networks}
Peer-to-peer communications whithin the network. Mobiles handle themself: \begin{itemize} \item control tasks \item networking tasks \end{itemize}
Advantages: robusness, node redundancy, lack of single points of failure, flexibility. \\
Applications: military, data networks, smart homes, device networks, (body) sensor networks. \\

\chapter{Ad-Hoc Wireless Networks}
Wireless sensors are widely used in telemedicine application, especially for connecting different sensors used to monitoring people in specific environment. An ad-hoc wireless networks is a collection of wireless mobile nodes that configure themselves to form a network without the aid of any established infrastructure.
Peer to peer communication within the network(), the terminal handle themselves:
\begin{itemize}
	\item control tasks
	\item networking task (how packets can be sent from one point to the other one)
\end{itemize}
Generally these tasks are implemented in a distributed way in order to have many advantages: robustness, node redundancy, lack of single points of failure, low costs for deployment and maintenance, flexibility. (if a node shut down the routing table will change in order to avoid failure in transmission)

\section{Applications}
They are used in military applications, data networks, smart homes, device networks, body sensory networks. A Self-configuring and lack of infrastructure, multi hop routing and distributed control result also in performance penalty in fact we have a decreasing in data rates and an increasing in delays.
Ad-hoc means specifically tailored on the application designer must predict type and prevention solution for cyber attacks

We can divide data networks in
\begin{itemize}
\item LAN
\item MAN
\item WAN
\end{itemize}
In their wired version we have good performance low cost and mostly device to device communication. We can also have the wireless verion of the previous which brings advantages in flexibility, reduced costs of deployment and maintenance, peer to peer communication, multi-hop through intermediate nodes could be useful. But introduce also challenges in maintaing high data rates, a link quality at each hope and an effective multihop scheme

\section{Home Networks}
Can be used for assisted living (with eventual remote control):
-smart rooms  (sense people and movement and adjust light and heating)
-monitoring system (coordinate and interpret data and alert to send to police/fire dept)

We have to define important parameters for network design that are \textbf{QoS}(quality of service) typically they regard directly data rates and delay constrains. They need standardization and vary in base of the specific application the network is used for. 

\section{Device Networks}
It means that we have a network composed by several devices placed in short range between each other, so we need to have protocol to ensure communication among this devices. This kind of network need low power and have low costs (devices usually count on radio)

\section{Sensor Networks}
In this networks we have typically small sensors and, in case of body area networks, we also need that they are enough small that a person forget to wear them. 
The signals that they acquire typically are high correlated each other, this give robustness to the network. 
They require little human intervention and they support movement or unmanned robotic vehicles. The most challenge is the battery life cause the battery cannot be so large and typically not rechargeable. (A possible solution is to try to harvest energy from the environment so from the human body itself)

Application could be:
\begin{itemize}
\item home environment: monitoring and regulation
\item large accidents or emergency situation so they could be used for example during rescue operation
\end{itemize}

\section{Distributed control systems}
Currently we have a centralized architecture with synchronous clock system and fixed topology. In ad-hoc systems we can increase robustness of network deploying larger number of nodes, have a distributed architecture and a flexible topology at the cost of rate and delays.

\section{Design principles}
The most important design is the peer to peer communication, this type of communication use also multi-hop strategies to send information. Compare to other wireless system in ad-hoc network we can form infrastructure in a permanent way or in a dynamic way instead on only the first of current wireless eg some nodes can perform as base stations.
Regarding the quality and the performance the most important measure is the SINR to measure the quality of the link, it is $ \propto\frac{1}{d^2} $ and depends on signal propagation and interference environment. It determines the data rate and the BER that can be supported by the link. In order to maintain a good SINR and depending on different topology of the network we need to adjust the transmit power form each sensors to the other.
Also the link parameters are variable, we can have a single hop communications where a code can send packet to its final destination directly, we can also have multiple-hop where packets are delivered form source to destination using intermediate relay nodes. In the second case we have a loss of power but facilitates scalability and decentralized control. The current challenges in a multihop network are to support high data rates and low delay cause we lose some power.

The scalability is the use of distributed network control algorithm which adjust local performance to account for local conditions. A typical trade-off during the information analysis is between local and centralized processing since it is an energy consuming process and, as we know, we need to save energy at every single sensor. another thing to take count during the transmit of data is that the  measured informations by sensors could be correlated so we could aggregate them and transmit only the most important data. The sleep mode can save energy of sensors but at the same time we need an efficient strategy to use it.

\section{protocol layers}
The first framework of layer was define by OSI where we had seven layers:application, presentation, session, transport, network, data link control, physical. With the internet coming we have reduced it to 5 layers: application, transport, network access and physical [image from slides] 
Is important to notice that some feature are present in more then one layer so we joint need to optimize parameters eg power control. 

\subsection{Physical layer}
The goal is to transmit bits over a point to point wireless link, the choice that are made in this layer influence also the layers above, it interact between PER, antennas and power control. It affects local neighbourhood\footnote{given a transmit power coupled with adaptive modulation and coding for a given node defines the collection of node that a node can reach with a single hop}.

\subsection{Access layer}
The goal is to deal with multiple/random access. We need also here some policy of  power control taking into account all the nodes in order to ensure a certain amount of SINR and giving resources to nodes (assigning channel or denying access). In case we have k node and N links:
	$$\gamma_k=\frac{g_{kk}P_k}{n_k+\rho\sum_{j\ne k}g_{kj}P_j}$$
where $ g_kj>0 $ is the channel power hain from the transmitter of j-th link to receiver of k-th link, $ P_k $ power of trnamitter on k-th link, $ n_k $ noise power of receiver of k-th link, $ \rho $ is the interference reduction due to signal proecessing (CDMA $ \rho\sim\frac{1}{G} $).\\
We can implement some decision algorithm to choose what power the node should transmit to other nodes, one parameter for each node. The SINR constraints for the network is given by: 
	$$ (I-F)P\ge u $$
with $ P>0 $ where $ P=(P_1,\dots,P_N)^T $ is the vector of N transmit power and $ u=(\frac{n_1\gamma_1^*}{denominatore},\dots) $.\\
Is possible to find an optimal solution
$$P^*=(I-F)^{-1}u$$
where $ F $ is a matrix and $ F_{kj}=\begin{cases}0& \text{with } k=j\\ \frac{\gamma_k^*g_{kj}\rho}{g_{kk}} &\text{with } k\ne j\end{cases} $.\\
When the SINR is below a certain target the power increases so it decreases when it is up that target\footnote{The information about the link has to be know only by the transmitter}. Cause the power is SINR varying we can have problems when we have non-static channels because the SINR changes and so the power of transmission. We need to take into account also for admission of new node to enter the system, when it happens the system has to update the vector of transmitting power of every single node.\\
We also have to deal with delay constraints: power control must be coordinated for ensure min delay on the end to end route. In this level, for ensure a more reliable connectione, is used retransmission(ARQ) and are added diversity bits that add redundancy to the packet in order to retrieve information when there are errors.

\subsection{Network layer}
The goal is to establish and maintain end-to-end connections in the network. The theoretically requirement is to have a fully connected network (different from graph theory cause a link could be multi-hop)
It deals with:
\begin{itemize}
	\item neighbour discovery.
	At first we have to initialize the network with randomly distributed nodes and some initial power. We have also requirements that have to be satisfied (minimum BER, $ P_{max} $ and minimum number of links ). If $ P_{max} $ and/or N small we have small disjoint clusters and if $ p_max $ and/or N large we need high power
	
	In order to create topology we can have a fully connected network we can decide some sophisticated distributed power control algorithm. Typically in case of static conditions(only path loss) we can have six to eight nodes and with mobility(fading) we have large variations of the conditions and is difficult to cope with instantaneous changes of fading. We can also have network diversity (if data is tolerant of delay) the same receiver could receive data from different transmitters. The topology is influenced by the PHY. 
	
	\item routing 
	How to send packet from one node to other nodes and is especially challenging when we have node mobility because routes need to be dynamically reconfigured and there are connectivity changes. There are three main category:
	\begin{itemize}
		\item floating: one packet to all the node in the receiving range and all the nodes broadcast the packet until the destination is reach Pros: robustness and little overhead Cons: we waste power and battery
		\item centralized the strategy is to determine information about the channel condition and from local network to send the information to the centre then the centre compute the best route for all the nodes of the network that send the routing tables to nodes. These tables are determined by optimization objectives: 
		\begin{itemize}
			\item min average delay
			\item cost for each hops
			\item Bellman-Ford or Dijkstra
		\end{itemize}
		with this strategy we can achieve an optimal routing but we have a very high overhead and difficult to adjust to fast changes
		
		\item the most common technique is distributed routing in which nodes send their connectivity info to neighbouring nodes and routes are computed from this local info pro minimal overhead and quickly adapts to link cons global routes sub-optima and routing loops are common
	\end{itemize}
	\item dynamic resource	
\end{itemize}
\chapter{Body Sensor Network}
\section{Bluetooth}
The technology came from an association from different company led by Ericsson (SIG - Special Interest Group). The name came from the danish king Harold I whose tooth was colored in war due to blueberries and the symbol came from the initials of King Harold ("K" and "H"). 

\subsection{Protocol}
It was bulit for short range communicaton (1-100 m) and it should be enerygy efficient as the purpose is to use it for portable devices. Seven workin groups participated on this huge project:
\begin{itemize}
\item G.1 AIr interface: Bluetooth radio(carrier-generation, modulation coding, power control)
\item G2 Software: Aim to make the protocol low-power based
\item G.3 Interoperability: Define user profiles for communication with other devices
\end{itemize}
There are some requirement consideration as:
\begin{itemize}
\item Transmission rates up to 1 Mbps
\item Energy efficiency: battery energy saving when data is not tx
\item No infrastructure is needed
\item Size should be mall for radio devices. For instance we have small nodes that are important for monitoring people movements
\item Utilization cost: it uses an unlicensed 2.4 GHz ISM band so we don't have to pay to use this frequency band
\item Monetary cost: not so high, quite cheap
\end{itemize}

\paragraph{Physical layer}
There are three power classes:
\begin{enumerate}
\item Class 1: Short range (1m), low power (1mW )
\item Medium range (10m) and power close to 4mW
\item Higher range (100m) and power equal to 100mW
\end{enumerate}
There are 3 basic configuration/functionalities
\begin{itemize}
\item Basic Rate (BR)
\item Enhanced data rate (EDR)
\item Low Energy (LE)
\end{itemize}

\paragraph{Channel access}
How can nodes communcicate? We have a frequency-hopping/time-division duplex scheme (FH-TDD) meaning We have a frequency-hopping spread spectrum and a time divsion scheme. Rx and Tx shared the frequency and at each transmission slot we have to have tx and rx syncrohnized.

Available bandwidth is spitted into $N:b$ sub-bands and the overall bandwisth is cose to 83.5MHz whereas the numebr of sub-bands or channels are 79 for USA and 23 for Japand. 
The carrier frequency can ben 2.402.GHZ + kMHz for k = 0,1,..78. Remeber that at each time slot we have a carrier frequency specified.

Bluetooth adopt an AFH that avoid overlapping with other non-hopping ISM systems that means that we skip some frequencies. 
\begin{itemize}
\item Guard itnervals: 2.2.5 MHz
\item Hopping rate: 1600 hops/s so that the time between hops (slot duration) is 625 microseconds
\end{itemize}

\paragraph{COmmunication modes}
COmmunciation modes can be 
\begin{enumerate}
\item Point to point
: 1 master, 1 slave. Master sets hopping sequence and cock, slave sycnronizes (clock and hopping sequence)
\item Point-to-multipoint (piconet): 1 master, up to 7 slaves. Master sets hopping sequence followed by all slaves
\item Multipoint-to-Multipoint (scatternet):
\begin{itemize}
\item connect multiple piconets
\item Three ways t interconnect piconets 
\begin{itemize}
\item Master-to-Master if masters are neighbors (one of them becomes slave of the other)
\item Slave as gateways
\item Intermediate gateways if two slaves of two masters are neighbors. New piconets include the two slaves 
\end{itemize}
\item DIfferent piconets have different frequency hopping scheme
\end{itemize}
\end{enumerate}
Piconet formation is made by a 2-step procedure:
\begin{enumerate}
\item Device discovery: master inquires device
\item Link establishment: negotation phase
\end{enumerate}

The start topology implies that slave-slave communciaton can go through only the master observation. 

\paragraph{Device discovery}
Master sends inquiring packs to the device, asking user to identify themselves and provide info about their clock. Thre's also the pre-defined inquiry hopping sequence and it is a 312.5 micorsecond sequence (twice hopping rate).

At the same time the candidate slave carries its own frequency carrier and it is very slow (1..28second) so it adopt to the master's one very slowsly. 
And that point, when they're syncronized, master and candidate slave cna meet and slave sends back to master a frequency-hopping sncronization pck (FHS) with ifo about its system clock. 

Inquirer invites a new node to hjoin the piconet (paging) and then the master sends another FHS with frequency hipping sequence thatallow a new slave to adopt to it. 

\paragraph{Link Layer}
Strict schedule of transmistting packet: time axis is divided into slot with boundaries set by the master because master set the clock information, the clocking sequence and hte boundaries of the slots. 
Slave suncrnized on boundariees. Once they're syncornized slave and master can communciate and exchange packs.

Packets transmission is performed thorugh time division duplex: we can have a single slot packet (at most one packets in each transmission slot). Master start transmitting at \emph{even} numberred tx slots while slave do that in \emph{odd} ones. 

Single-slot pc are 27 bytes of data. One pck every 2 slots tha timplies a step of 1250 microsecond. Transmission rate is172.8 kbps (unidirection, uplink or downlink). Multi-slot packet can be 183 (three slots) or 339 (five slots) bytes and the transmission rates can be585.6kbps or 723.2 kbps. 

With multi-slot pck the carrier frquency is kept unchanged during the transmission because we need to have a stable communication. 

\paragraph{Transmission modes}
\begin{enumerate}
\item Sychronous connection-oriented (SCO): the amster assign the trasnsmission slot to the slaves periodically. This can b a good idea for data flow at regualr itnervals
\item Asynchronous connectionless (ACL) link: master polls a given slave (ask a slave to send infos at some poit) whenever an event occurs, e.g master has to send data to it.
\end{enumerate}
These 2 kind of linsk could be use together at the same time so the master can allocated idiffernet can of links for different slve s. 

ARQ scheme is used in Bluetooth when we have some error and it can help to have retransmission of corrupted packets. It goes from master to the slaves or viceversa. 

\paragraph{Piconet scheduling}
In the piconet we can have differen t scheduling scheme 
\begin{itemize}
\item Pure Round Robin (PRR): each slave is polled one/cycle (cycle = time taken for serving all slaves). the Max delay is bounded, but waste of resources in case no data are available at some slaves
\item Limited Round Robin: as PRR, multiple slots could be allocated for each slav. Increased throughput but still resources waste. 
\item Exhaustive Round Robin: slaves are polled until they have data to send. Further increase of throughput but possible unfairness
\item Adaptive Round Robin: unloaded slaves could be skipped. Needs for priority and traffic control schemes. Possible delays for unloaded users. 
\end{itemize}

\paragraph{Power Modes}
\begin{enumerate}
\item Active: slave always on and listening to all transmission slots. Max responsiveness, but highest power consumption. Nodes are all the time active
\item Sniff: Slave periodically active at regular intervals with the agree with the master. SLave checks for polling message by the maser, otherwise puts rdio to sleep. Slaves must keep synchronization with the master and the rest of the tine they can save energy, but to do so they're less responsive in case of unscheduled polling
\item Hold: slave \textbf{previously} agree with the master to stop listening to channel for specific amount of time  s the nodes can save even more energy compare to the sniff modes. 
\item Park: Nodes that are not in active/sniff/hold states that are no longer considered as piconet members. They still have to maintain synchronism with the master and periodically listen to it. If they want to transmit again they have to perform another phase of exchanging information as it were the first communication between them two. This kind of nodes are called \emph{parked users}. Energy savings is good and each piconet can have at most 1 master, 7 members and 25 parked users.  
\item Standby: user no longer needs to maintain synchronism and negotiation phase is required before starting a new communication. 
\item Other energy saving mechanisms implemented int he Bluetooth as for a slave to ask the master to adjust transmission power in order to keep at minimum error-free communication. 
\end{enumerate}

\paragraph{Open Issues}
\begin{itemize}
\item Automatic network formation is not supported a
\item If the master moves away, network collapses so  the dynamic of the network should change by this
\item Starting up new connection requires some time like 5s (quite high interval) due to channel and page scanning
\item While a node is in Inquiry mode, if there was some transmission ongoing it will be interrupted by the inquiring set up
\item Inquiry fails if both devices are in inquiry state
\item Communication between piconet (scatternet) is defined mostly as proprietary solutions (no standards)
\item Support for efficient multicast is missing
\end{itemize}

\subsection{Bluetooth LW Energy - BLE}
This represent an implementation of the Bluetooth has some common feature with the main protocol still performing other differences. As in the Bluetooth there is the implementation of device discovery, connection establishment, connection mechanisms with the purpose of having low energy solution and cheaper solutions for portable device. 

Main feature can be represented as follow:
\begin{itemize}
\item 2.4GHz ISM, 40RF channels
\item Bandwidth: 2 MHz
\item Spread spectrum: AFH
\item Multiple access: FDMA and TDMA
\begin{itemize}
\item FDMA: 40 channels = 3 advertising + 37 data
\item TDMA: polling-based data tx (master-slaves)
\item Piconet (master determines hop interval and hopping pattern for 37 channels)
\end{itemize}
\item 2 Phy channels:
\begin{enumerate}
\item LE advertising: is a broadcast channel for discovery, interaction of devices prior connection and the communication could be establish with every number of devices. This has a short duration though. 
\item LE piconet channel: it perform communication between connected devices, active physical link is a master-slave communication link (we can't perform scatternet communication as in Bluetooth) and the access is controlled by the master. Slaves cannot have > 1 active links with master   
\end{enumerate}
\end{itemize}
\paragraph{Access}
Insert figure
\paragraph{Logical Topology}
\begin{itemize}
\item \textcolor{red}{'Gna faccio più raga}
\item Slaves connected on same master do not share physical channel 
\end{itemize}
So BLE represents an energy-efficient alternative to original standard

















\chapter{1D Bio Signals}
--Slide reading 1-13 --

The different muscle fibers produces combinations of signal deriving from the motor unit neuronal connection. This gives a triphasic signal (MUAP superposition)

Slide reading 15-end

\section{Biosignal compression}
LECTURE 18/12/17---------------------------------------------------------------------\\
Standard compression (slides 2-3-4):
\begin{itemize}
  \item SPIHT: Originally developed for images, now used also for 1D bio signals. Principle: DWT (Discrete Wavelet Transform) coefficients in different subbands have temporal relationship with one another. This compression algorithms allows us to have direct control on the compression ratio, since in any moment we can stop bit stream.
  \item JPEG2000: both loss-less and lossy compression. Three steps: DWT, then quantization of coefficients and finally arithmetic coder (from samples to bitstream).
\end{itemize}
Performance evaluation (slide 5): Compression Ratio (CR) and Percentage Root-mean-squared Distortion (PDR).\\
For ECG (slides 6-7):
\begin{itemize}
  \item[1.] the most important thing is to identify the QRS complex $\rightarrow$ threshold on amplitude, neural net, wavelets, ...
  \item[2.] segmentation: divide the signal into parts of equal length (quasi periodic signal)
  \item[3.] time domain compression techniques: AZTEC, CORTES, LTC
  \item[4.] frequency domain: FFT, DCT, DWT (high computational complexity)
  \item[5.] also parametric techniques: neural net, vector quantization, CS, pattern matching, denoising autoencoders (quite new, seems to give good compression performance with a reasonable computational complexity)
\end{itemize}
Other signals (slide 8).\\
Compressed sensing of EEG (slide 11): a lot of literature. A good advantage is detected when > 22 electrodes (a lot). \\
How much can we compress the EEG? (slide 12 for numbers found) also some literature, both range of acceptable PRD levels and PSNR. Test for different reconstruction algorithms (slides 13).
Compressed sensing of ECG (slide 14-15-16-17-18): also many recent papers.\\
Other biosignals: EOG, EMG (slide 19).\\
Compressed sensing in BAN + smartphone/server for continuous monitoring (slide 20).\\
Conclusions (slide 21-22): tipically compressed sensing has been tried on different databased and compared, CS performs better for ECG compared to EEG, however it deprends con CR. The main challenge is still in the reconstruction.\\

% \input{chapter_PP.tex}
% \input{renewalProc.tex}
% \input{gbn_analysis.tex}
% \input{chapter_SN.tex}

\end{document}
