\chapter{Electroencephalography}
The EEG signal comes from the most exterior part of the brain and it's an electrical signal. The outer part of the brain contains the synapsis, wich increases whith aging of the person.
[slide 3]
The neural network can be tree-based and if all neurons in a tree are activated at the same time, we get the EEG signal.

[Slide 4]
The EEG propagation can be modeled as
\begin{equation}
  \underbrace{EEG}_{P_rc}= \underbrace{LPF}_{P_{TX}} + \underbrace{propEffects}_{shadow/fading} + \underbrace{artefacts}_{noise}
\end{equation}

In fact, putting an electrode in a point we get the signal coming in the "line of sight" and the propagation of other signals which are transmitted via the skull and the scalp, but aren't in the LOS.

\section{Brain activity characterization}\label{sec:bac}
In order to study the EEG signal, we need to take into account the activity of the brain in the \begin{itemize}
  \item Time domain
  \item Frequency domain
  \item Space domain (the topology)
\end{itemize}
\subsection{Time domain}
[slide omonima]
In the time domain we have that the power in the movement situation (figure bottom left) of the signal is that it starts to become constant, wherease in the resting state, there are three main lobes.

\subsection{Frequency domain}
The frequency domain of the EEG goes in the interval $[0.01;500] Hz$, but EEG machines, usually, don't go under 1.4 Hz and above 60 Hz
To receive other frequency we need to use other techniques.
[slide rhythms]
The activity of the brain can be divided into different types of frequencies. The amplitude is reversly proportional to the frequency, so with high frequency the signal is almost distrupted by path loss and shadowing.

\subsection{Space domain}
We know that some region are used by the brain for specific activities. Not completely true, butgeneral tendency. In the slides there are the somatosensoty strip and motor strip. In the motor strip, if we move the hand, it's easier to get signals (most reactive region of the brain) in the brain when hand movement is done.

The homunculus plot we have some part of the body given which area is excited when something is done. The today area is not to use the functional approach, but the connectivity/circuit approach: brain seems to be more likely to have circuits to enable some activities. To accomplish some functions we have some circuits activating together to accomplish that function.

23/10/2017

\section{Acquisition of EEG}
We revidse the comparison between an EEG signal and a wireless transmitted signal:
\begin{esp}\label{eq:eeg-wireless}
  EEG &= LPF + \text{propagation effects} + \text{artifacts}\\
  P_{RC} &= P_{TX} + \text{path loss/shadowing} + \text{noise}\\
\end{esp}
where LFP Local Field Potential

\subsection{The inverse problem}
The electrode is a signal source which sends some test signal $s_{TX} = \delta(t)$ to test the received signal into another electrode. The path loss is proportional to $e^{-d}$ and introduces a delay $\tau_0$. We know that there's a kind of pathloss as in the model shown in \eqref{eq:eeg-wireless}

If we set more than one source transmitting a known signal (e.g. $\delta(t)$), we can collect all the delay thanks to the sensors and we can get a matrix which describes the propagation effects (delay, attenuation, cross attenuation) from each source to the sensors.

Inverting that matrix leads to the characterization of the brain. The problem is that we don't actually have a known transmittable signal without other ones, so we need to rely on models known. A reliable source model is an open problem.

We need to characterize the signal as said in \autoref{sec:bac}. We have brain (system) with the neurons. The communication of neurons happens in groups of neuron. Each group can be part of the same circuit (motor, perception/touch, sight, acoustic). There are some circuits interacting with each other creating the system. The visual circuits and the motor one are well studied and well known; other are less known. Now there are studies on why and how the brain ``stop'' of some circuits if other are working. For example Moser studied neurons that activates while the individual is handling space navigation (movements). Same set of neurons works only in some specific frequencies. Neurons can code signals in time, frequency and in phase.

\subsubsection{Sensori-motor circuit}
The cortex and basal ganglia communicates with some circuit. There are some interesting **
for example, if we want to move the arm, the cortex (basal ganglia) process the tactile info of the hand (e.g. resting on a table) and uses this information, transmitted with a specific frequency, that will be received and processed to specify a movement. The reception of the signal is slow (f=6Hz) The updating of a movement is quite faster instead (f>20Hz).

As the received power is inversly proportional to the frequency, the brain need to put more energy to high-frequency signal, but brain is constantly in power-save mode, so uses the higher frequency to propagate the signal in a smaller space, wherease the lower frequency for long-distance signals.

The neurons need to be synchronized and this is a topic that's being studied. The synchronization is used to lower the transmission energy, as the two groups are ready to transmit/receive without interfering.

\subsubsection{Circuit modeling}
One useful and needed approach is the multi-modal: to see the same system we record data from different sensors (EEG, magnetic information, oxigen consumption).
Another option is stimulating the brain to test the circuit. It's not possible to send a signal and record it from the sensor. We though use some non-invasive techniques to approximate the test signal. An example is the TMS (transcranial magnetic stimulation); another example is the muscular stimulation: we let the individual do some actions and we activate some regions. We can record the signal that was produced.

\section{Recording}
When we record the EEG we need to set the montage of the electrods and the the cap, which sets where to put the sensors. Importants are two references: the INION (Iz) on the back of the head and the Nasion, the front of the brain. The electrods are made analyzing the left and right preauricular distance to set the electrodes with a specified proportion. To make it easier, there's a standard cap.

The signal is acquired using a differential amplifier, which makes the difference between the reference and the signal.

The standard montage has 10-20 systems which corresponds to 200 electrodes approximately. The sampling needs a frequency at least fo 256Hz with 12/24 bit to represent the levels. Some systems are wired, other are wireless. The portable version is appreciated as less intrusive.
\subsection{Paradigm}
To record the EEG signal we typically need to design the paradigm: typically we don't simply put the cap and record, but we use a paradigm to have a better signal. At the beginning of the session, the doctor tells the patient what he will do while the acquisition is on, the stimuli given to the patient. A different stimuli produces different response of the brain, so the recording can be falsified and almost-unreadable as the signal can be corrupted. So not knowing the state of the person can lead to an useless signal.
the acquisition goes as follows
 Typically the person does nothing when the screen is blank. when something else is shown, the person enters in a warning state, then it receives some other stimulus and does some actions for a fixed amount of time and then goes to rest. This sequence is called \textbf{trial}. In order to compensate variability of the brain, the trial is repeated.

 We can measure the resting state to analyze the brain activity related to the memory dealing with the environment stimulous. In case of stimuli, we can analyze the event-related potentials (event=stimulus) and event-related (de)synchronization of the signal.

 \subsection{Artifacts}
 When we record the EEG we have 2 types of artifacts:
 \begin{itemize}
   \item Powerline noise given by the coupling noise of the amplifier
   \item electrode displacement, given by the misplacing of an electrode, the touch of an electrode and the TMS, for example
 \end{itemize}
 Muscular activity (around the face): swallowing, eye closing, mouth movement/chewing. Each signal has a frequency which is specific (eye ~~ 3Hz, chewing ~~ 10Hz) and the signal is spatial specific. Alpha waves overlaps, for example, with eye movement, so the signal need to be cleaned from the eye artifacts.

 The pre-processing phase consists in clean the signal from artifacts, enhance SNR, some temporal filters and spatial filters, which exploits the info given by different sensors in different positions to enhance the SNR.

 Sometimes we use the interpolation, we surely segments the signal and it's useful to decompose the signal by separating the frequencies that compose it (PCA and ICA).

 We need to select which processing of the signal. Surely the pre processing explots filtering, segmentation and one of PCA or ICA.

 When we stimulate, there's a drop of the signal (fast, 1s) and it takes a lot to recover the signal (10s) that may be discarded. Some informations are transmitted in 0.1 s, so 10s is really a lot of wasted signal that can though be used if appropriate filtering is applied.

 One criticality is*

 which can lead to the non-stimulation of the patient and waste time.

 One solution is the zero-padding when the signal drops and filtering the really low component, to be able to recover the signal without wasting it.
 High frequency signal (teeth clanching - scritto male-) can be eliminated with LPF.

 One spatial filter to suppress noise is the laplacian. The small laplacian is used if electrodes filtered are near, Large laplacian if they're far away. To get a higher SNR a temperature sensor can be useful.

 We can use a low pass filter or high pass filter to attenuate some frequencies. Other useful filters are the notch, which eliminates a specific frequency, or a bandpass filter to consider only a band of frequencies.

 \subsection{Segmentation}
 The acquired signal is divided into smaller time signals if the signal is non-stationary, so to make it stationary. Typical length is 2s (deterministic). On the segment, we can decide automatically or by-eye to discard a signal if it's too bad. On each segmen we can compute some statistical properties such as the amplitude distributions (standard deviation, kurtosis, skewness). For example, we can discard a signal that has a standard deviation higher than a threshold as it's considered too noisy. To eliminate trials with some artifacts we can set a threshold that, if exceeded, set the signal as ``bad''

 \subsection{Independent Component Analysis (ICA)}
 Different signals which spatial derivation is known, can be mapped into another signal space to map it to another signal space using matrix decomposition. In this new signal space, ICA can identify artifacts and delete them (e.g. artifacts related to muscular movements)
 We can then discards some of the components of the whole signal. Removal of the components and going back to the original signal space, we now have a signal with higher SNR.

 \section{EEG signal processing}
 \begin{enumerate}
   \item DC correction
   \item  spatial filter 1 to remove the reference sensor effect (intracephalic)
   \item temporal filters
   \item trials extraction
   ...
 \end{enumerate}

 We need to understand if, in the processing phase, we want to derive from the clean dataset of signal into a dataset of sources, more closely related to the brain.

 A famous algorithm is the LORETA.

 The signal can be analyzed with the fourier transform, feature extraction and classification, done with machine learning algorithms.

\subsection{Measures}
From the EEG we can measure different things:
activity (resting, event related)
connectivity measures: measures that quantify the relation of activities at different points. We investigate coherence, to find similarity of the signal int he frequency domain. Cross frequency to find

we can be interested into the relation between brain activity and muscular activity at the same time.

\subsection{ERP}
The typical waveform is shown in figure. The analysis is done in the time-domain, typically averaging the segments taken in a identical same particular train phase.

We identify, at different times (100ms interval), the response of the brain with an alternance of local maxima and minima of the signals.

Every peak has a meaning: for example the p300 is made out of 2 near peaks.
The first peak, usually is the receiption of the signal from the ear, eye, touch and the second as the reaction of the signal.

The paradigm to get the erp is:
we need to set a set of stimuli, not important, followed by one stimulus that is important and the patient is focused on it, followed by non-important stimuli. Then the trial is repeated and averaged to get the final signal.

For neurologist, two important peaks are the N200 and the P300, which can determine some cognitive problems.

24/10/2017 lect \# 9

\subsection{Event Related Desynchronization}
Used to investigate the brain activity during the movements. For example, when performing a movement, we arrive from a situation of resting and after a stimuli, the person need to do a thing (e.g. grab a pen). Along with the EEG we can have the EMG contemporarily (boh). We record, in the emg the situation on the muscle where, after the sound, the muscle starts contracting and then keeps contrancting. The EEG (ERD) at the same time presents a phase where there's a resting power. After the stimuli, the power decreases and says at a low level until the movement is completed. This is a typical behaviour.

Where can we extract the signal? We cannot extract the signal if it's coming in the frontal or occipital part of the brain. instead we record from the center part of the brain, along the line from one ear to the other.

The frequencies band are $ \mu \approx \in [8;13]Hz ~~\beta \approx \in [13;30]Hz$ for $\mu$ and $\beta$ rithm. The subject increases the activity before the movement and mu waves during.

From the baseline of the signal, we usually have a peak and then a lower baseline. After the movement we have the baseline of before-movement (figura 1 gimp)

The mu signal is strong in $C_3$, wherease the $\beta$ signal can be deirved on both the $C_3$ and $C_z$.

Sometimes we can extract the lines of the EEG from the EMG and the knowledge on the time when the stimulus was applied. From the two plots we can derive the brain period of the subject related to the movement.

\subsubsection{Algorithm to extract $\mu$ and $\beta$ bands}

The algorithm starts averaging the power, then comparing to a model. Usually the beta band is negative until the movement ended.

Another visualization is the power of the signal on each receiver put on the brain. This can show where the desynchronzation happens and usually starts from c3 and after a while we get the signal on the opposite side c4 (ipsilateral signal) as showing that the right part of the brain controls the left part of the body.

The time-frequency decomposition of the signal shows the variation of the signal power in time at different frequencies, so we can show the desynchronization, in blue, which is not self substained (boh), but it's an itra-individual characteristic.

We can have the same plot if the person is just imagining the movement, without doing it. The trial for this sequence is the same as the one with the actual movement (rest, warning, imagination).

This can be useful to analyze the movement and the strength of a movement to extend it to people that can only imagine it.

\subsubsection{hyppocampus studies [ hot topic]}
The cross frequencies coupling is quite useful to understand the condition of a person and find if there are some problems in accomplishing actions.
The hyppocmamous structure is made out of substructures and its well known. The substructures communicates with each other using known signal. The carrier [4;8]Hz for the transmission from the source to the receiver is in the theta band and it's based on 2 carriers modulated. The receiving area is activated using a different carrier with a different phase  (e.g. slow carrier. at a certain phase, we have another area sending another signal sent at a higher frequency (gamma band))
$\theta \in [4;8]Hz ~~ \gamma > 30Hz (\text{human}) \, or \, >60Hz$

The gamma activity happens at a specific point of the phase of theta, for example a typical behaviour is the one where the gamma activity happens on $\theta$' trough. On the top of the $\theta$ modulation there's another signal (modulated with AM), the $\gamma$ is modulated inside the $\theta$ modulated signal. The principal signal is sent at theta band wherease the feedback signal is sent, synchronously, at the gamma band modulated with the theta modulation. This is the Cross-frequency coupling.


An animal study on mouse is studied when the mouse need to store new memori (e.g. new path). We see a burst of beta/theta (boh) activity.

\subsection{CFC for synchronization}
The brain-muscle model has a sensor that from the muscle tx the signal to the thalamus and the basal ganglia and the motor area retx the signal to the muscle.

Synchronization is required for the movement, so more than one signal need to be transmitted at the same time when other signals are also transmitted. To keep energy consumption low, lower frequencies are used for long-range comm, wherease higher frequencies( faster communication) are used for shord and rapid signal, so we can get the meaning (muscle or inter-brain communication) of the signal also considering its frequency.

slide analog modulation cfc, self explainatory.

We then need to demodulate the AM signal, to recover the useful signal. To demodulate, we apply the well known demodulation theory:

The signal has the cross-frequency noise modulated with some pink noise. Pink noise has $\P_\mu(f) \propto \frac{1}{f}$. With standard demodulation algorithm, we loose a lot of useful signal, while getting the main theta signal.

Using the AM theory we can get the theta signal and the gamma one, so x-freq coupling is preserved.
------
cortico-muscular coherence slide

We can imagine to have x,y two signals, one recorded with EMG, the other with EEG. After a certain number of trials, we can find the correlation between pairs and we may want to get the maximum of the correlation function. We may be interested on the lag, which means when, in time, the two signal have maximum correlatio. With this we can find the time for the signal to be received by the muscle when transmitted by the brain. The leg time is aproximatively about 10/20 ms, which is about the propagation delay of the signal.

To analyze the coherence we need the two pss and the cross PSD. The MSC finds the carrier that transmits a specific signal as it has the same frequency in both the brain and the muscle.

MEG=MagnetoEncephaloGram

The coherence has a peak in the beta range, so that frequency band could be a good candidate to explain the communication between brain and muscle.

The approaches we saw were based on old techniques. nowadays there's the Machine Learning / deep network approach to find features from a lot of datas. The features are no more the cross-coupling, but there are some new, artificial, features. The two approaches to analyze the signals are called \emph{featuring} and \emph{architectural} approaches.

example hands motor imagery. The neural network algorithm has an accuracy, increasing the data set, approximatively about 86.57\%.

The architectural approach has some gaps with the phisiological data as we cannot know if it's finding some new features that we didn't know or if the algorithm is acting badly.

\subsection{Abnormal EEG}
EEG is usually useful for findind epilepsy, where amplitude is abnormal, but also for other brain malfunctions, like Parkinson or stroke diseases where activity is exaggerated on some frequency bands.

NIRS Near InfraRed Spectroscopy is non invasive and, like the fMRI, analyzes the oxigenation of the brain.
