\chapter{Medical Image Processing}
Medical imaging:
\begin{itemize}
  \item MRI: non invasing technology, the principle behind it is to use a strong magnetic field to align the protons spin, then a preselected pulse/signal is given to align all the protons to this new direction, when they go back to the previous direction the scanner is able to dected the energy produced. We can acquire anatomical info.
  \item Functional MRI: to acquire info about the functions processed (expecially for the brain). Compare baseline (subject doing nothing) with data acquired giving the subject a specific task to perform. Requires very long acquisition time, usually for research purposes.
  \item Tractography.
  \item CT: computerized X-ray imaging. Invasive technique, radiation to the subject.
  \item PET: also invasive, radioactive tracers.
  \item Sonography: ultra-sound to get the image of hhuman body, not invasive, achieves a good visualization of the subcutaneous body structures. 2 technologies: transmission (now abandoned) and reflection (echo technologies).
  \item PLI: polarized light imaging.
  \item FM: fluorescence microscopy.
\end{itemize}
